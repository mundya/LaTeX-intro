% Define the title and author, these will be inserted when we use the
% `\maketitle` command.
\title{An Introduction to \LaTeX{}}
\author{Andrew Mundy}
\date{}  % I don't really want to add the date

% Set Source Code Pro as the typewriter style font
\usepackage{sourcecodepro}

% The Minted package is for syntax highlighting of code samples
\usepackage{minted}
\setminted[tex]{
  fontsize=\small,
  frame=lines,
  linenos=True,
  style=friendly,
  xleftmargin=2em,
  xrightmargin=2em,
}

% This package allows us to include graphics
\usepackage{graphicx}
\usepackage[export]{adjustbox}  % Allows use to use the `frame` option

% This package is for pretty tables
\usepackage{booktabs}

% Begin the document (this ends the preamble)
\begin{document}
  % Add the title, in presentation mode this will add the title slide, in
  % article mode this will just add a title.
  \maketitle

  % Sections are created with the `\section{Section Title}` command, I'm going
  % to be mixing slides in with normal written content to avoid creating two
  % completely separate documents.
  \section{Introduction}
  \subsection{Some disclaimers}
  \subsubsection{Tools}

  Choosing the tools to perform a task can be quite a subjective issue and sadly leads to most significant flame-wars\footnote{I'm writing this in \texttt{vim}.}.
  As writing is such a significant part of our work choosing the right tool for writing with is particularly important.
  Its worth noting that while I use LaTeX for most of my academic writing I occasionally use other tools (e.g., Google Docs) when they're useful.

  \subsubsection{Other tutorials}

  This tutorial is based Andrew Roberts' excellent set of LaTeX tutorials\footnote{\texttt{http://andy-roberts.co.uk/writing/latex/}}.
  I've removed content that I feel is misleading (\texttt{eqnarray} and some table formatting) and tried to provide references to current best practises for achieving certain results.

  \subsection{Why use LaTeX?}

  \section{Getting Started}
  % - Introduce the important parts of a LaTeX document: the preamble and the
  %   actual content
  % - Introduce some simple commands
  %   - Bullet points
  %   - Enumeration

  LaTeX documents are composed of a series of \texttt{.tex} files which are compiled into a finished document by calling some appropriate command.
  Which command will depend on what you intend to achieve but my preference is for using \texttt{pdflatex} -- this will take a \texttt{tex} file and produce a PDF with no extra mucking around required.
  Other tools can produce PostScript or DVI files if required.
  This process is significantly different from Microsoft Word and friends, where what we type into the editor is exactly what we get in the end; instead producing LaTeX documents is more like compiling code.

  The code below is pretty much the smallest LaTeX document that we could produce.

  \inputminted{tex}{examples/example-1.tex}

  This is included in the examples directory for this document and can be made into a PDF by executing the following in the shell.

  \begin{minted}{bash}
    pdflatex example-1
  \end{minted}

  Figure~\ref{fig:intro/example1} shows what we get when we build the document (this is clipped to avoid scaling an entire US Letter sized document).

  \begin{figure}[t]
    \centering  % The text in this environment is to be centred
    \includegraphics[trim=5cm 20cm 5cm 5.25cm, clip, frame]{examples/example-1}
    \caption{Our very first LaTeX document compiled}
    \label{fig:intro/example1}
  \end{figure}

  Already we've encountered some important ideas.
  Everything between the start of the file and \mintinline{tex}{\begin{document}} is called the \textit{preamble}.
  The preamble should contain all of the code for importing packages, creating new commands and setting settings.
  Everything between \mintinline{tex}{\begin{document}} and \mintinline{tex}{\end{document}} is said to be in the \texttt{document} \textit{enviroment} and will be included in the typeset output.
  We'll see some more environments later.

  Going through line by line:
  In the first line we specified the type of document we wish to create, in this case an article.
  We do this by calling \mintinline{tex}{\documentclass{article}}.
  There are some other document classes that we might want to use, and some publishers may require that specific classes are used.

  \begin{table}[h!]
    \centering
    \begin{tabular}{l l}
      \toprule
      Class & Use \\
      \midrule
      \texttt{acmsmall} & Some ACM publications. \\
      \texttt{article} & Documents like this. \\
      \texttt{beamer} & Presentations. \\
      \texttt{IEEETran} & IEEE journal and conference presentations. \\
      \texttt{report} & Longer technical documents (think end of year report\ldots or thesis!). \\
      \bottomrule
    \end{tabular}
  \end{table}

  Lines 2 and 3 specified the title and author of the document.
  If we'd wanted we could have added a subtitle or the date.
  These lines aren't typeset immediately but when we call \mintinline{tex}{\maketitle} they're used to generate a title which is appropriate for our document class, for example a title page or slide.
  Line 5 starts the document and line 6 prints the title.
  Line 8 is the body text that is typeset (the \mintinline{tex}{\LaTeX{}} command just prints \LaTeX{}).
  Finally, line 9 ends the document.

  \section{Document Structure}
  % - Introduce the structuring of LaTeX documents
  %   + Hints for making large document manageable

  \section{Mathematics}
  % - Introduce equation environments
  % - Progress through typesetting increasingly complex expressions
  %   - Multiline
  %   - Matrices
  %   - Pointers to extra symbol libraries, extra fonts

  \section{Tables}
  % - Simple tables with booktabs
  % - More complex tables with siunitx

  \section{Images and Graphics}
  % - Including graphics
  % - Floats and captions

  \section{Bibliographies}
  % - Citing stuff
  % - Creating a bibliography
  % - Recommendations

  \section{Tables of Contents}
  % - Creating tables of contents

  \section{Presentations}
  % - Introduction to Beamer
  % - Introduction to the Unofficial University of Manchester theme

  \section{Tips for large documents and source control}
  \subsection{Large documents}
  % - Splitting sections up

  \subsection{Source control}
  % - Sections per line for source control
  %   - Additional Git commands

  \subsection{Working with \texttt{IEEETran}}
  % - Dredge up fixes for IEEETran

  \section{Creating diagrams directly in LaTeX with TikZ}
  % - Brief introduction to TikZ
  % - Point at the documentation - it IS excellent
\end{document}
